\documentclass[11pt, letterpaper]{article}

% --- PACKAGES ---
\usepackage[utf8]{inputenc} % Handle UTF-8 input encoding
\usepackage[T1]{fontenc}    % Use modern font encoding
\usepackage{geometry}       % For setting page margins
\usepackage{amsmath}        % For math environments and symbols
\usepackage{amssymb}        % For additional math symbols
\usepackage[
    colorlinks=true,        % Enable colored links
    linkcolor=blue,         % Color for internal links
    filecolor=magenta,      % Color for file links
    urlcolor=cyan,          % Color for URL links
    pdftitle={Final Project Report}, % PDF metadata
    pdfauthor={Adrian Hoang},        % PDF metadata
    pdfpagemode=FullScreen  % PDF view mode
]{hyperref}                 % For clickable links and PDF metadata
\usepackage{booktabs}       % For professional quality tables (\toprule, \midrule, \bottomrule)
\usepackage{caption}        % For table captions
\usepackage{datetime2}      % Use for \today command formatting if needed
% \usepackage{array}          % No longer strictly needed with tabularx for this table
\usepackage{tabularx}       % *** ADDED for better table formatting ***
\usepackage{ragged2e}       % *** ADDED for better text alignment in X column ***

% --- GEOMETRY ---
\geometry{letterpaper, margin=1in} % Standard 1-inch margins on letter paper

% --- DOCUMENT START ---
\begin{document}

% --- Improved Summary Table using tabularx ---
\begin{table}[htbp] % h: here, t: top, b: bottom, p: page -- lets LaTeX decide placement
 \centering
 \caption{Evaluation Summary: LLM vs. SMT Verdicts}
 \label{tab:summary}
 % Use tabularx to fit the text width. X column allows text wrapping.
 % l = left align, c = center align, X = paragraph wrapping, ragged right
 \begin{tabularx}{\textwidth}{@{} l >{\RaggedRight}X c c l @{}}
  \toprule
  \textbf{Case ID} & \textbf{Fact Pattern Summary}                 & \textbf{Model Output} & \textbf{SMT Verdict} & \textbf{Status} \\
  \midrule
  Case 1  & Marcus \& Lily (23yo QC Student, >50\% support, low income)           & Yes          & Yes         & Correct \\ \addlinespace % Add a little extra space between rows
  Case 2  & Ellen \& Zack (17yo Nephew, fails residency, fails self-support, high income)            & Yes          & No          & Error   \\ \addlinespace
  Case 3  & Raj \& Meena (68yo Mother, QR, low income, valid MSA assumed)                 & Yes          & Yes         & Correct \\ \addlinespace
  Case 4  & Luis \& Paco (35yo Cousin/HH Member, >50\% support, fails QR income)         & No           & No          & Correct \\ \addlinespace
  Case 5  & Helen \& Irene (90yo Mother, QR, low income, ambiguous/verbal MSA)        & Yes          & No          & Ambiguous   \\ \addlinespace
  Case 6  & Brenda \& Carl (85yo Father, QR, low income, ambiguous support level) & No           & No          & Ambiguous \\
  \bottomrule
 \end{tabularx}
\end{table}

\end{document}
